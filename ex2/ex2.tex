\documentclass{article}
\usepackage[utf8]{inputenc}
\usepackage{amssymb}
\usepackage{amsfonts}
\usepackage{amsmath}
\usepackage[utf8]{inputenc}
\usepackage[norsk]{babel}
\usepackage{lmodern}
\usepackage{float}
\usepackage{graphicx}
\DeclareGraphicsExtensions{.pdf,.png,.jpg}
\usepackage{latexsym}
\usepackage{gensymb}
\usepackage{hyperref}
\usepackage[euler]{textgreek}
\usepackage{stackengine}
\usepackage[version-1-compatibility]{siunitx}
\usepackage{pdfpages}
\usepackage{fixltx2e}
\hypersetup{pdfborder={0 0 0}}
\usepackage{pgfplots}
\pgfplotsset{compat=newest}
\pgfplotsset{plot coordinates/math parser=false}
\newlength\figureheight
\newlength\figurewidth
\usepackage[section]{placeins}%Sørger for at plots og andre floats holder seg til sin section.
\setlength\parindent{0pt}%Setter indent til 0



\title{Exercise 2 - TTK4130 Modeling and Simulation}
\author{Camilla Sterud}
\date{}

\begin{document}

\maketitle

\newpage

\section*{Problem 2}

There are three criteria that ust be fullfilled by the rational, proper transfer funtion $H(s)$ for it to  a positive real:

\begin{enumerate}
\item All the poles of $H(s)$ have $Re(\lambda_i) \leq 0$.
\item \label{crit:two} $Re[H(j\omega)] \geq 0$ $\forall$  $\omega$ s.t. $j\omega$ is not a pole of $H(s)$.
\item If $j\omega_0$ is a pole of $H(s)$, it is simple and $Res_{s = j\omega_0}[H(s)] > 0$.
\end{enumerate}


\subsection*{a}

\begin{equation*}
	H_1(s) = \frac{1}{1 + Ts}.
\end{equation*}

$H_1(s)$ has a pole at $-\frac{1}{T}$, which has a neagtive real part for $T > 0$.

\begin{equation*}
    Re[H_1(j\omega)] = Re[\frac{1}{1 + Tj\omega}] = \frac{1}{1 + \omega^2T^2} \geq 0 \quad \forall \: \omega.
\end{equation*}

$\underline{\underline{H_1(s) \, \textrm{is positive real}}}$

\begin{equation*}
    H_2(s) = \frac{s}{s^2 + \omega_0^2}.
\end{equation*}

$H_2(s)$ has a pair of complex conjugated poles at $\pm j\omega_0$, which has a real part of zero. If we picture the phase plot of $H_2(j\omega)$, it will start out in $90\degree$ because of the zero in $\omega = 0$. The complex conjugate pole pair will cause the phase to fall by $180\degree$, making the phase end up at $-90\degree$. This means that $H_2(s)$ will always stay in the right half plane of the complex plane and we can conclude that $Re[H(j\omega)] \geq 0$.

\begin{equation*}
    Res_{s = j\omega_0}[H_2(s)] = \lim_{s\to j\omega_0}(s - j\omega_0)H_2(s) = \lim_{s\to j\omega_0}\frac{s}{s + j\omega_0} = \frac{1}{2} > 0.
\end{equation*}

$\underline{\underline{H_2(s) \, \textrm{is positive real}}}$


\subsection*{b}

\begin{equation*}
    H_3(s) = \frac{s+a}{(s + b)(s + c)}, \quad b, c > 0.
\end{equation*}

$H_3$ has two ploes in $-b$ and $-c$, both with negative real parts. 

We have several cases in need of consideration to decide for which a $H_3(s)$ is positive real. The poles in $-b$ and $-c$ will contribute to the phase with $-180 \degree$. If $a < 0$, $H_3(s)$ will start out in the left half plane, so that can be excluded. If $a > b,c$ the phase will fall below $-90 \degree$ before $a$ can pull it up again by $90\degree$. The only cases where $H_3(s)$ stays in the right half plane is if $a = 0$, thus stating out the phase plot in $90\degree$, or if $a < b + c$. 

This can also be seen by calculating the real part of $H_3(s)$:

\begin{equation*}
    Re[H_3(j\omega)] = \frac{abc + \omega^2(b+c-a)}{(\omega^2 + b^2)(\omega^2 + c^2)}.
\end{equation*}


It is easy to see that this will only always be positive if the numerator stays positive for all $\omega$, as the denominator will always be positive. Thsi only happens in the same cases as mentioned above, so

$\underline{\underline{H_3(s) \, \textrm{is positive real for } 0 \leq a < b + c}}$

\subsection*{c}

\begin{equation*}
    H_4(s) = \frac{s^2 + a^2}{s(s^2 + \omega_0^2)}, \quad a \geq 0.
\end{equation*}

The poles of $H_4(s)$ are at $0$ and $\pm j\omega_0$, which are all at the imaginary axis with zero real part. 

\begin{equation*}
    Re[H_4(j\omega)] = Re[\frac{a^2 - \omega^2}{j\omega(\omega_0^2 - \omega^2)}] = 0,
\end{equation*}

so criterion \ref{crit:two} is fullfilled. 

\begin{equation*}
    Res_{s = 0}[H_4(s)] = \lim_{s \to 0} s \frac{s^2 + a^2}{s(s^2 + \omega_0^2)} = \frac{a^2}{\omega_0^2} > 0 \quad \forall \: a \neq 0.
\end{equation*}

\begin{equation*}
    Res_{s = j\omega_0}[H_4(s)] = \lim_{s \to j\omega_0} (s-j\omega_0) \frac{s^2 + a^2}{s(s^2 + \omega_0^2)} = \frac{\omega_0^2 - a^2}{2\omega_0^2} > 0, \quad a \in (-\omega_0,\omega_0)
\end{equation*}

$\underline{\underline{H_4(s) \, \textrm{is positive real for } 0 < |a| < |\omega_0|}}$

\subsection*{d}

\begin{equation*}
    T\dot y = -y + u \Rightarrow f(y,u) = \frac{1}{T}(-y + u)
\end{equation*}

\begin{equation*}
    \dot V = \frac{\partial V}{\partial y}f(y,u) = u^Ty - g(y) \quad \forall \: u, \quad g(y) > 0.
\end{equation*}

\begin{equation*}
    V = \frac{1}{2}Ty^2
\end{equation*}

\begin{equation*}
    \dot V = Ty\dot y = y(-y + u)
\end{equation*}


\begin{equation*}
    \dot V = uy - y^2 \Rightarrow g(y) = y^2 > 0 \Rightarrow Passive
\end{equation*}



\end{document}