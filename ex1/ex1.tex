\documentclass{article}
\usepackage[utf8]{inputenc}
\usepackage{amssymb}
\usepackage{amsfonts}
\usepackage{amsmath}
\usepackage[utf8]{inputenc}
\usepackage[norsk]{babel}
\usepackage{lmodern}
\usepackage{float}
\usepackage{graphicx}
\usepackage{latexsym}
\usepackage{hyperref}
\usepackage[euler]{textgreek}
\usepackage{stackengine}
\usepackage{fixltx2e}
\hypersetup{pdfborder={0 0 0}}
\usepackage{pgfplots}
\pgfplotsset{compat=newest}
\pgfplotsset{plot coordinates/math parser=false}
\newlength\figureheight
\newlength\figurewidth
\usepackage[section]{placeins}%Sørger for at plots og andre floats holder seg til sin section.
\setlength\parindent{0pt}%Setter indent til 0



\title{Exercise 1 - TTK4130 Modeling and Simulation}
\author{Camilla Sterud}

\begin{document}

\maketitle

\newpage

\section{Problem 1}

\begin{align}
	Ni = \phi(\mathcal R_a + \mathcal R_c + \mathcal R_b + \mathcal R_r).\label{eq:Ni}\\ 
	\mathcal R_a = \frac{z}{A\mu_0}, \mathcal R_r = const. \label{eq:R_a} \\ 
	\mathcal R_c, \mathcal R_b << \mathcal R_r, \mathcal R_a. \label{eq:small}
\end{align}


\subsection{a}

\begin{equation}\label{eq:R_r}
	\mathcal R_r = \frac{z_0}{A\mu_0}, z_0 = const.
\end{equation}

Using Equation \ref{eq:Ni} together with the relations from equations \ref{eq:R_a} and \ref{eq:R_r} we get

\begin{equation*}
	Ni = \phi(\frac{z}{A\mu_0} + \mathcal R_c + \mathcal R_b + \frac{z_0}{A\mu_0}).
\end{equation*}

Since $\mathcal R_c$ and $\mathcal R_b$ are neligible (Equation \ref{eq:small}), the total magnetomotive force on the ball is

\begin{equation*}
	\underline{\underline{Ni = \frac{\phi}{A\mu_0}(z + z_0).}}
\end{equation*}

\subsection{b}

\begin{equation} \label{eq:induct}
	L(z) = \frac{N\phi}{i} = \frac{N^2A\mu_0}{z + z_0}.
\end{equation}

\begin{equation}\label{eq:magnF}
	F = \frac{i^2}{2}\frac{\partial L(z)}{\partial z}.
\end{equation}

Assume positive direction downwards and gravitational acceleration $g$.

\begin{align*}
	ma = \Sigma F\\
	m\ddot z = mg + F\\
	m\ddot z = mg + \frac{i^2}{2}N^2A\mu_0(z + z_0)^{-2}.
\end{align*}

\begin{equation}\label{eq:motion}
	\underline{\underline{\ddot z = g - \frac{1}{2m}i^2N^2A\mu_0(z+z_0)^{-2}}}
\end{equation}


\subsection{c}
Linearizing about $z_d$, $\dot z_d = 0$, $\ddot z_d = 0$.
\begin{align*}
	0 = g - \frac{1}{2m}i_d^2N^2A\mu_0(z_d+z_0)^{-2}\\
	i_d = \frac{1}{N}\sqrt{\frac{2mg}{A\mu_0}}(z_d + z_0)
\end{align*}

Equation \ref{eq:motion} we now call $f_1$. We define $z = z_d + \Delta z, i = i_d + \Delta i$ and thereby $\dot z = \Delta\dot z$. A Linearization of Equation \ref{eq:motion} around the point $z_d$ is then

\begin{align}
	\Delta\ddot z = \left. \frac{\partial f_1}{\partial \dot z} \right|_{z = z_d}
\end{align}


\end{document}