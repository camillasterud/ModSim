\documentclass{article}
\usepackage[utf8]{inputenc}
\usepackage{amssymb}
\usepackage{amsfonts}
\usepackage{amsmath}
\usepackage[utf8]{inputenc}
\usepackage[norsk]{babel}
\usepackage{lmodern}
\usepackage{float}
\usepackage{graphicx}
\DeclareGraphicsExtensions{.eps,.pdf,.png,.jpg}
\usepackage{latexsym}
\usepackage{hyperref}
\usepackage[euler]{textgreek}
\usepackage{stackengine}
\usepackage{listings}
\usepackage[version-1-compatibility]{siunitx}
\usepackage[usenames,dvipsnames]{color}
\usepackage{pdfpages}
\hypersetup{pdfborder={0 0 0}}
\usepackage{pgfplots}
\pgfplotsset{compat=newest}
\pgfplotsset{plot coordinates/math parser=false}
\newlength\figureheight
\newlength\figurewidth
\usepackage[section]{placeins}%Sørger for at plots og andre floats holder seg til sin section.
\lstset{flexiblecolumns=true}


\lstloadlanguages{Matlab}
\setlength\parindent{0pt}%Setter indent til 0
\definecolor{MyDarkGreen}{rgb}{0.0,0.4,0.0}


\lstloadlanguages{Matlab}%
\lstset{language=Matlab,                        % Use MATLAB
        frame=single,                           % Single frame around code
        basicstyle=\small\ttfamily,             % Use small true type font
        keywordstyle=[1]\color{Blue}\bfseries,        % MATLAB functions bold and blue
        keywordstyle=[2]\color{Purple},         % MATLAB function arguments purple
        keywordstyle=[3]\color{Blue}\underbar,  % User functions underlined and blue
        identifierstyle=,                       % Nothing special about identifiers
                                                % Comments small dark green courier
        commentstyle=\usefont{T1}{pcr}{m}{sl}\color{MyDarkGreen}\small,
        stringstyle=\color{Purple},             % Strings are purple
        showstringspaces=false,                 % Don't put marks in string spaces
        tabsize=5,                              % 5 spaces per tab
        %
        %%% Put standard MATLAB functions not included in the default
        %%% language here
        morekeywords={xlim,ylim,var,alpha,factorial,poissrnd,normpdf,normcdf},
        %
        %%% Put MATLAB function parameters here
        morekeywords=[2]{on, off, interp},
        %
        %%% Put user defined functions here
        morekeywords=[3]{FindESS, homework_example},
        %
        morecomment=[l][\color{Blue}]{...},     % Line continuation (...) like blue comment
        numbers=left,                           % Line numbers on left
        firstnumber=1,                          % Line numbers start with line 1
        numberstyle=\tiny\color{Blue},          % Line numbers are blue
        stepnumber=5                            % Line numbers go in steps of 5
        }

% Includes a MATLAB script.
% The first parameter is the label, which also is the name of the script
%   without the .m.
% The second parameter is the optional caption.
\newcommand{\matlabscript}[2]{\begin{itemize}\item[]\lstinputlisting[caption=#2,label=#1]{#1.m}\end{itemize}}


\title{Exercise 3 - TTK4130 Modeling and Simulation}
\author{Camilla Sterud}
\date{}

\begin{document}

\maketitle

\newpage

\section{Problem 1}

\subsection{a}

For spool 1:

\begin{equation*}
\underline{\underline{\Sigma\tau = J_1\dot\omega_1 = F_1r_1 - T_1 + B_1\omega_1}}
\end{equation*}

For spool 2:

\begin{equation*}
\underline{\underline{\Sigma\tau = J_2\dot\omega_2 = -F_2r_2 + T_2 + B_2\omega_2}}
\end{equation*}


\subsection{c}

Chose right as positive direction. 

\begin{align*}
	\Sigma F &= F_2 - F_1\\
	\dot x &= v_2 - v_1
\end{align*}

\begin{equation*}
	\underline{\underline{F_2 - F_1 = Kx^2 + B\dot x}}
\end{equation*}

\subsection{d}

Know that 

\begin{align*}
	F_1v_1 - F_2v_2 &= F_kv_k \\
	F_k &= kx_k, \quad \dot x = v_k,
\end{align*}

and from the figure in the exercise we see that

\begin{align*}
	F_k &= F_1\cos(\theta) + F_2\cos(\theta)\\
	F_1\sin(\theta) &= F_2\sin(\theta).
\end{align*}	

From this it is easy to derive that 

\begin{align*}
	v_1 - v_2 &= 2v_k\cos(\theta)\\
	F_1 &= F_2 = \frac{F_k}{2\cos(\theta)}
\end{align*}

\subsection{g}

\begin{figure}[!ht]\label{fig:bode}
	\centering
	\includegraphics[width = 0.8\textwidth]{modsim_ex5_bode}
    \caption{The bode plot for the system. The code for generating this plot can be seen in Listing \ref{modsim_ex5}.}
\end{figure}

From the bode plot in Figure \ref{fig:bode}, we see that the magnitude is the same for the two belts, but that the response of the second belt is phase delayed by $\SI{180}{\degree}$. 


\matlabscript{modsim_ex5}{Code for making the Bode plot in MATLAB}



\end{document}